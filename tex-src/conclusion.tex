%% Макрос для заключения. Совместим со старым стилевиком.
\startconclusionpage

В данном разделе размещается заключение.

\section{Выводы}

В рамках работы не удалось решить задачу 2-d сегментации по причине неприменимости подобных 2d end-to-end методов к датасету LIDC-IDRI. Это еще более подтвердило уместность задачи локализации. Заимствованную модель для решения задачи локализации удалось обучить на кропах меньшего размера и получить относительно неплохие показатели чувствительности.

Оптимизированная архитектура GAN, предложенная в работе \cite{mirsky}, показала свою способность улучшить качество локализатора и порождать правдоподобные и разнообразные опухоли посредством добавления Wasserstein Loss и модуля Adain.

\section{Применимость полученных методов на современных данных}

Современные томографы обладабт более высоким разрешением, нежели сканы, представленые в использованных датасетах. Пространственное разрешение может быть более, чем в 4 раза выше по каждой оси, что ведет к огромному весу полных 3d сканов. Однако предложенные методы способны обрабатывать входные данные неограниченных размеров, и модель локализации при необходимости можно обучить на входных данных большего размера

Существует несколько способов решения этой проблемы:

\begin{enumerate}
    \item Уменьшение размерности КТ изображений
    \item Обучение модели локализации и аугментации на данных большего разрешения. Возомжно при достаточных вычислительных мощностях
\end{enumerate}