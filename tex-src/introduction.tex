
\section{Актуальность}

Из всех раковых заболеваний рак легких наиболее рапространен по всему миру. Более того
на данный момент он является одним из самых тяжелых раковых заболеваний по показателям смертности. В связи с этим задача распознавания новообразований по КТ изображениям легких очень актуальна, поскольку эффективные методы поиска новообразований способствуют диагностике заболеваний на ранней стадии, что позволяет своевременно назначить лечение и в конечном счете сократить число летальных исходов.

\section{Описание данных}

Трехмерные сканы легких, полученные с помощью компьютерной томографии могут могут содержать до 512 * 512 * 600 вокселов. При этом в основном диаметр опухоли составляет не более 30mm (среднее значение диаметра новообразований в датасете LIDC-IDRI \cite{lidc} равняется 15 мм). Каждый воксел в среднем эквивалентен от 0.7 до 2 мм по каждой из пространственных осей, что свидетельствует о том, что опухоли невелики по сравнению со всем изображением.

\section{Цели и задачи}

Целью работы является разработка методов и программных средств автоматического распознавания новообразований в легких на изображениях компьютерной томографии. 

Среди задач можно выделить два подхода, альтернативных друг другу

\begin{enumerate}
    \item Сегментация двухмерных изображений для определения маски, соответствующей опухоли
    \item Локализация трехмерного учатска, содержащего новообразование, на трехмерном изображении.
\end{enumerate}

Отметим, что вторая задача является существенно более распространенной несмотря на то, что
привычным методом поиска новообразований является сегментация изображений. В данном же случае локализация подходит лучше, поскольку опухоли достаточно компактны и малы, и естественно предсказывать их наличие и местоположение с помощью bounding box.

При рассмотрении второй задачи была сформулирована еще одна подзадача, которая может существенно помочь в решении задачи локализации, - аугментация данных. Размеры используемых датасетов невелики, несмотря на их громоздкость, и генерация искуственных данных может существенно улучшить качество работы моделей локализации.

Эффективность использования аугментации данных с помощью GAN подтверждена несколькими работами в том числе \cite{han2019synthesizing} \cite{wgan-augmentation-2} \cite{wgan-augmentation}
