
Из всех раковых заболеваний рак легких наиболее рапространен по всему миру. Более того на данный момент он является одним из самых тяжелых раковых заболеваний по показателям смертности. В связи с этим задача распознавания новообразований по КТ изображениям легких очень актуальна, поскольку эффективные методы поиска новообразований способствуют диагностике заболеваний на ранней стадии, что позволяет своевременно назначить лечение и в конечном счете сократить число летальных исходов.

Трехмерные сканы легких, полученные с помощью компьютерной томографии могут могут содержать до $512 \times 512 \times 600$ вокселов. При этом в основном диаметр опухоли составляет не более 30mm (среднее значение диаметра новообразований в датасете LIDC-IDRI \cite{lidc} равняется 15 мм). Каждый воксел в среднем эквивалентен от 0.7 до 2 мм по каждой из пространственных осей, что свидетельствует о том, что опухоли невелики по сравнению со всем изображением. 

Маленькие размеры опухолей, отсутствие некоторого 

