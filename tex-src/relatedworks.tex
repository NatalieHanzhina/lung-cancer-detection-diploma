\chapter{Обзор современных результатов в области распознавания новообразований в легких}

%% Так помечается начало обзора.
\startrelatedwork

\section{Сверточные нейронные сети} 

  

Сверточные нейронные сети являются одной из основных технологией в компьютерном зрении и глубоком обучении. Появление сверточных сетей наряду с графическими процессорами (GPU), рассчитанными на массивные параллельные вычисления, позволило совершить огромный прорыв в области решения задач классификации, локализации, сегментации и т.д. 

  

\subsection{ResNet (Residual Neural Network, Остаточная нейронная сеть)} 

  

ResNet --- это популярная сверточная сеть, предложенная в работе \cite{he2016deep}. Основной особенностью являются skip свзязи --- переходы между слоями, не являющимися соседними к друг другу и помогающие в борьбе с проблемой затухающего градиента. 

  

\subsection{Кодировщик-декодировщик (Encoder-Decoder)} 

  

Encoder-Decoder является распространенной архитектурой нейронных сетей, подразумевающий наличие кодировщика и декодировщика. Кодировщик --- это последовательность слоев сети, которые преобразуют вход, существенно сокращая его размерность. На выходе кодировщика получается некоторый тензор из латентного пространства, который должен эффективно представлять вход сети, используя для этого данные меньше размерности. К выходу кодировщика применяется декодировщик, который в свою очередь преобразует латентное представление входа сети в некоторые данные большей размерности. 

  

\subsection{Region Proposal Network (RPN)} 

  

Впервые RPN была предложена в работе \cite{ren2015faster} как часть сети Faster-R-CNN и предназначается для решения задачи локализации. RPN надстраивается над выходом сверточной сети, содержащем feature map исходного изображения. Помимо наличия основного входа RPN параметризуется также и набором анкоров - участков изображения, построенных на основании размерностей исходного изображения. Для каждого из анкоров RPN необходимо подать на выход предсказание, насколько этот анкор совпадает с локализируемыми участками, и набор несколько преобразованных координат анкора. 

  

\subsection{Squeeze and Excitation (SE)} 

  

Модуль SE был предложен в 2017 году в работе \cite{hu2018squeeze} и показал SOTA результат в соревновании ImageNet. Данный модуль, который предполагается включать в сеть после каждого сверточного блока, позволяет использовать зависимости между различными каналами блока и масштабировать с помощью вектора коэффициентов, обучаемых в небольшой побочной сети. 

  

\section{Генеративные противоборствующие сети} 

  

Генеративные противоборствующие сети эффективно используются не только для генерации реалистичных фотографий, лиц и т.д., которые можно было бы использовать вместо настоящих фотографий для некоторых конечных целей, но также широко применяются для аугментации данных, которые предполагается использовать для обучения. 

  

Основная идея GAN состоит в создании двух независимых сетей --- генератора и дискриминатора, и задачей генератора является генерация реалистичных объектов по некоторому входу из латентного пространства, а задачей дискриминатора --- классификация объектов на генерированные и оригинальные. 

  

Обучение GAN представляет из себя минимаксную игру, где генератор стремится увеличить функцию потерь дискриминатора, а дискриминатор --- напротив, уменьшить ее. 

  

Также выделяют отдельный класс генеративных противоборствующих сетей --- Deep Convolutional GAN (DCGAN), которые используют глубокие сверточные сети в качестве моделей генератора и дискриминатора. 

  

\subsection{Условные генеративно-состязательные сети (Conditional GAN, CGAN)} 

  

В данной вариации GAN на вход генератору подаются не только данные из латентного пространства, но и некоторые данные условия (condition). Таким образом, генератору приходится подстраивать генерируемые данные под это условие, чтобы они выглядели реалистично. 

  

\subsection{Wasserstein GAN (WGAN) \cite{arjovsky2017wasserstein}} 

  

Данная модификация GAN заключается в том, что вместо бинарного классификатора дискриминатор выдает вещественные числа (чем больше, тем больше уверенность в оригинальности изображения). Характерной особенностью WGAN является функция потерь - Wasserstein Loss. 

  

\begin{equation} 

    L(y_{true}, y_{pred}) = mean(y_{true}y_{pred}) 

\end{equation} 

  

  

\section{Адаптивная нормализация объектов} 

  

Адаптивная нормализация объектов --- технология нормализации выходов внутренних слоев сети, которая была предложена в работе \cite{huang2017arbitrary} для совершенствования модели переноса стиля одного изображения на другое 

  

Это аффинное преобразование входа $x$ параметризованное некоторым $y$: 

  

\begin{equation} 

AdaIN(x, y) = \sigma(y)(\dfrac{x - \mu(x)}{\sigma(x)}) + \mu(y) \label{eq:adain} 

\end{equation} 

  

Авторы предполагали использовать стиль в качестве $y$, однако в последствии данную нормализацию стали применять. 

  

В области генеративных противоборствующих сетей AdaIN можно применять не только для решения задачи переноса стиля, но и для обучения эффективных параметров нормализации. В роли $y$ может выступать выход некоторой побочной сети, принимающей на вход латентный вектор. 

  

\section{Существующие решения} 

  

Рассмотрим некоторые существующие решения задач локализации и аугментации данных. 

  

\subsection{Локализация} 

  

Множество моделей локализации используют двухэтапную схему \cite{girshick2014rich} на подобии следующей:  

  

\begin{enumerate} 

    \item Получения region proposals из входного изображения и вычисление признаков для каждого из proposal-ов с помощью 3D-CNN 

    \item Классификация участков (в нашем случае на содержащие и не содержащие опухоль) 

\end{enumerate} 

  

Основной причиной выбора такой двухэтапной схемы является большое количество ложно положительных предсказаний, получающихся в end-to-end методах, которые возможно устранить на втором этапе подобной схемы (FP reduction). 

  

Другой класс моделей использует end-to-end 3D-CNN для решения задачи локализации. В частности популярна сеть 3D-Faster-RCNN, включающая с себя RPN. 

  

\subsection{Аугментация} 

  

В ряде работ предложена аугментация датасетов с помощью. В частности в работе \cite{wgan-augmentation} использовался WGAN, а в работе \cite{han2019synthesizing} использовался Multi-Conditional GAN. В качестве модели локализации (для обучения которой и необходима аугментация) использовалась Faster-RCNN, а в качестве GAN --- MCGAN c 4-мя сверточными слоями в кодировщике и 4-мя в декодировщике, а также skip-связями. Дискриминатор по своей структуре похож на Pix2Pix GAN, а в качестве функции потерь использовался Wasserstein Loss (с Gradient Penalty) наряду с Least Squares Loss. 

  

\section{Постановка цели и задач ВКР} 

  

Целью работы является разработка методов и программных средств автоматического распознавания новообразований в легких на изображениях компьютерной томографии.  

  

Среди задач можно выделить два подхода, альтернативных друг другу 

  

\begin{enumerate} 

    \item Сегментация двухмерных изображений для определения маски, соответствующей опухоли 

    \item Локализация трехмерного участка, содержащего новообразование, на трехмерном изображении. 

\end{enumerate} 

  

Отметим, что вторая задача является существенно более распространенной несмотря на то, что 

привычным методом поиска новообразований является сегментация изображений. В данном же случае локализация подходит лучше, поскольку опухоли достаточно компактны и малы, и естественно предсказывать их наличие и местоположение с помощью bounding box. 

  

При рассмотрении второй задачи была сформулирована еще одна подзадача, которая может существенно помочь в решении задачи локализации, --- аугментация данных. Размеры используемых датасетов невелики, несмотря на их громоздкость, и генерация искусственных данных может существенно улучшить качество работы моделей локализации. 

  

Эффективность использования аугментации данных с помощью GAN подтверждена несколькими работами в том числе \cite{han2019synthesizing} \cite{wgan-augmentation-2} \cite{wgan-augmentation} 


\section{Выводы по главе 1}

В данной главе приведены и кратко описаны основные базовые методы и технологии, заимствованные в работе или имеющие к ней отношение. Наряду с обзором предметной области в данной главе описаны цель и задачи, поставленные в данной работе, а так же их обоснование.

\finishrelatedwork