\documentclass[times,specification,annotation]{itmo-student-thesis}

%% Опции пакета:
%% - specification - если есть, генерируется задание, иначе не генерируется
%% - annotation - если есть, генерируется аннотация, иначе не генерируется
%% - times - делает все шрифтом Times New Roman, собирается с помощью xelatex
%% - languages={...} - устанавливает перечень используемых языков. По умолчанию это {english,russian}.
%%                     Последний из языков определяет текст основного документа.

%% Делает запятую в формулах более интеллектуальной, например:
%% $1,5x$ будет читаться как полтора икса, а не один запятая пять иксов.
%% Однако если написать $1, 5x$, то все будет как прежде.
\usepackage{icomma}

%% Один из пакетов, позволяющий делать таблицы на всю ширину текста.
\usepackage{tabularx}

%% Данные пакеты необязательны к использованию в бакалаврских/магистерских
%% Они нужны для иллюстративных целей
%% Начало
\usepackage{tikz}
\usetikzlibrary{arrows}


% @inproceedings{ example-english,
%     year        = {2015},
%     booktitle   = {Proceedings of IEEE Congress on Evolutionary Computation},
%     author      = {Maxim Buzdalov and Anatoly Shalyto},
%     title       = {Hard Test Generation for Augmenting Path Maximum Flow 
%                   Algorithms using Genetic Algorithms: Revisited},
%     pages       = {2121-2128},
%     langid      = {english}
% }

% @article{ example-russian,
%     author      = {Максим Викторович Буздалов},
%     title       = {Генерация тестов для олимпиадных задач по программированию 
%                   с использованием генетических алгоритмов},
%     journal     = {Научно-технический вестник {СПбГУ} {ИТМО}},
%     number      = {2(72)},
%     year        = {2011},
%     pages       = {72-77},
%     langid      = {russian}
% }

% @article{ unrestricted-jump-evco,
%     author      = {Maxim Buzdalov and Benjamin Doerr and Mikhail Kever},
%     title       = {The Unrestricted Black-Box Complexity of Jump Functions},
%     journal     = {Evolutionary Computation},
%     year        = {2016},
%     note        = {Accepted for publication},
%     langid      = {english}
% }

% @book{ bellman,
%     author      = {R. E. Bellman},
%     title       = {Dynamic Programming},
%     address     = {Princeton, NJ},
%     publisher   = {Princeton University Press},
%     numpages    = {342},
%     pagetotal   = {342},
%     year        = {1957},
%     langid      = {english}
% }


\begin{filecontents}{bachelor-thesis.bib}
@article{li2019deepseed,
  title={DeepSEED: 3D Squeeze-and-Excitation Encoder-Decoder ConvNets for Pulmonary Nodule Detection},
  author={Li, Yuemeng and Liu, Hangfan and Fan, Yong},
  journal={arXiv preprint arXiv:1904.03501},
  year={2019}
}

@online{ mirsky,
    title       = {CT-GAN: Malicious Tampering of 3D Medical Imagery using Deep Learning},
    author      = {Yisroel Mirsky and Tom Mahler},
    url         = {https://www.usenix.org/system/files/sec19-mirsky_0.pdf},
    year        = {2019},
    langid      = {english}
}

@inproceedings{huang2017arbitrary,
  title={Arbitrary style transfer in real-time with adaptive instance normalization},
  author={Huang, Xun and Belongie, Serge},
  booktitle={Proceedings of the IEEE International Conference on Computer Vision},
  pages={1501--1510},
  year={2017}
}


@online{ wgan-augmentation,
    title       = {Augmenting LIDC Dataset Using 3D Generative AdversarialNetworks to Improve Lung Nodule Detection},
    author      = {Chufan Gao, Stephan Clark et al.},
    url         = {https://www.researchgate.net/publication/331723419_Augmenting_LIDC_dataset_using_3D_generative_adversarial_networks_to_improve_lung_nodule_detection},
    year        = {2019},
    langid      = {english}
}

@inproceedings{han2019synthesizing,
  title={Synthesizing diverse lung nodules wherever massively: 3D multi-conditional GAN-based CT image augmentation for object detection},
  author={Han, Changhee and Kitamura, Yoshiro and Kudo, Akira and Ichinose, Akimichi and Rundo, Leonardo and Furukawa, Yujiro and Umemoto, Kazuki and Li, Yuanzhong and Nakayama, Hideki},
  booktitle={2019 International Conference on 3D Vision (3DV)},
  pages={729--737},
  year={2019},
  organization={IEEE}
}

@online{ wgan-augmentation-2,
    title       = {hz},
    author      = {hz},
    url         = {https://www.ncbi.nlm.nih.gov/pmc/articles/PMC6334309/},
    year        = {2019},
    langid      = {english}
}

@article{ lidc,
    author      = {Samuel G Armato III, et al.},
    title       = {The lung image database consortium (LIDC) and image database resource initiative (IDRI): a completed reference database of lung nodules on ct scans},
    journal     = {Medical Physics},
    number      = {vol. 38, no. 2},
    year        = {2011},
    pages       = {915-931},
    langid      = {english}
}

@article{ luna,
    author      = {Arnaud Arindra Adiyoso Setio, et al.},
    title       = {Validation, comparison, and combination of algorithms for automatic detection of pulmonary nodules in computed tomography images: the luna16 challenge},
    journal     = {Medical Image Analysis},
    number      = {vol. 42},
    year        = {2017},
    pages       = {1-13},
    langid      = {english}
}

@article{arjovsky2017wasserstein,
  title={Wasserstein gan},
  author={Arjovsky, Martin and Chintala, Soumith and Bottou, L{\'e}on},
  journal={arXiv preprint arXiv:1701.07875},
  year={2017}
}

@inproceedings{ren2015faster,
  title={Faster r-cnn: Towards real-time object detection with region proposal networks},
  author={Ren, Shaoqing and He, Kaiming and Girshick, Ross and Sun, Jian},
  booktitle={Advances in neural information processing systems},
  pages={91--99},
  year={2015}
}

@inproceedings{hu2018squeeze,
  title={Squeeze-and-excitation networks},
  author={Hu, Jie and Shen, Li and Sun, Gang},
  booktitle={Proceedings of the IEEE conference on computer vision and pattern recognition},
  pages={7132--7141},
  year={2018}
}

@inproceedings{he2016deep,
  title={Deep residual learning for image recognition},
  author={He, Kaiming and Zhang, Xiangyu and Ren, Shaoqing and Sun, Jian},
  booktitle={Proceedings of the IEEE conference on computer vision and pattern recognition},
  pages={770--778},
  year={2016}
}

\end{filecontents}
%% Конец

%% Указываем файл с библиографией.
\addbibresource{bachelor-thesis.bib}


\begin{document}

\studygroup{M3439}
\title{Распознавание новообразований по КТ изображениям легких}
\author{Крючков Максим Игоревич}{Крючков М.И.}
\supervisor{Фильченков Андрей Александрович}{Фильченков А.А.}{к.ф-м.н.}{Научный сотрудник Университета ИТМО}
\publishyear{2020}
%% Дата выдачи задания. Можно не указывать, тогда надо будет заполнить от руки.
\startdate{01}{сентября}{2019}
%% Срок сдачи студентом работы. Можно не указывать, тогда надо будет заполнить от руки.
\finishdate{31}{мая}{2020}
%% Дата защиты. Можно не указывать, тогда надо будет заполнить от руки.
\defencedate{15}{июня}{2020}

\addconsultant{Ханжина Н.Е.}{инженер ФИТиП}

\secretary{Павлова О.Н.}

%% Задание
%%% Техническое задание и исходные данные к работе
\technicalspec{Требуется разработать методы и программные средства автоматического распознавания новообразований в легких на изображениях компьютерной томографии.}

%%% Содержание выпускной квалификационной работы (перечень подлежащих разработке вопросов)
\plannedcontents{Описание предметной области и существующих решений задачи. Описание предложенных моделей решения задач локализации, сегментации и аугментации. Описание сложностей, возникших при решении задач. Описание экспериметальных результатов тестирования полученных моделей и сравнение с существующими решениями}

%%% Исходные материалы и пособия 

%%% Цель исследования
\researchaim{Разработка методы и программнык средства автоматического распознавания новообразований в легких на изображениях компьютерной томографии.}

%%% Задачи, решаемые в ВКР
\researchtargets{\begin{enumerate}
    \item Сегментация двухмерных изображений для определения маски, соответствующей опухоли;
    \item Локализация трехмерного учатска, содержащего новообразование, на трехмерном изображении;
    \item Аугментация данных с трехмерными изображениями КТ легких.
\end{enumerate}}

%%% Использование современных пакетов компьютерных программ и технологий
\addadvancedsoftware{Tensorflow}
% \addadvancedsoftware{Google Colaboratory}

%%% Краткая характеристика полученных результатов 
\researchsummary{Были получены отрицательные результаты решения первой задачи. Задача аугментация была успешно решена и улучшено качество модели локализации посредством обучения на аугментированном датасете}

%%% Гранты, полученные при выполнении работы 

%%% Наличие публикаций и выступлений на конференциях по теме выпускной работы
\researchpublications{По теме этой работы я ничего не публиковал.
\begin{refsection}
\printannobibliography
\end{refsection}
}

%% Эта команда генерирует титульный лист и аннотацию.
\maketitle{Бакалавр}

%% Оглавление
\tableofcontents

%% Макрос для введения. Совместим со старым стилевиком.
\startprefacepage


\section{Актуальность}

Из всех раковых заболеваний рак легких наиболее рапространен по всему миру. Более того
на данный момент он является одним из самых тяжелых раковых заболеваний по показателям смертности. В связи с этим задача распознавания новообразований по КТ изображениям легких очень актуальна, поскольку эффективные методы поиска новообразований способствуют диагностике заболеваний на ранней стадии, что позволяет своевременно назначить лечение и в конечном счете сократить число летальных исходов.

\section{Описание данных}

Трехмерные сканы легких, полученные с помощью компьютерной томографии могут могут содержать до 512 * 512 * 600 вокселов. При этом в основном диаметр опухоли составляет не более 30mm (среднее значение диаметра новообразований в датасете LIDC-IDRI \cite{lidc} равняется 15 мм). Каждый воксел в среднем эквивалентен от 0.7 до 2 мм по каждой из пространственных осей, что свидетельствует о том, что опухоли невелики по сравнению со всем изображением.

\section{Цели и задачи}

Целью работы является разработка методов и программных средств автоматического распознавания новообразований в легких на изображениях компьютерной томографии. 

Среди задач можно выделить два подхода, альтернативных друг другу

\begin{enumerate}
    \item Сегментация двухмерных изображений для определения маски, соответствующей опухоли
    \item Локализация трехмерного учатска, содержащего новообразование, на трехмерном изображении.
\end{enumerate}

Отметим, что вторая задача является существенно более распространенной несмотря на то, что
привычным методом поиска новообразований является сегментация изображений. В данном же случае локализация подходит лучше, поскольку опухоли достаточно компактны и малы, и естественно предсказывать их наличие и местоположение с помощью bounding box.

При рассмотрении второй задачи была сформулирована еще одна подзадача, которая может существенно помочь в решении задачи локализации, - аугментация данных. Размеры используемых датасетов невелики, несмотря на их громоздкость, и генерация искуственных данных может существенно улучшить качество работы моделей локализации.

Эффективность использования аугментации данных с помощью GAN подтверждена несколькими работами в том числе \cite{han2019synthesizing} \cite{wgan-augmentation-2} \cite{wgan-augmentation}


%% Начало содержательной части.

\chapter{Обзор современных результатов в области распознавания новообразований в легких}

%% Так помечается начало обзора.
\startrelatedwork

\section{Сверточные нейронные сети} 

  

Сверточные нейронные сети являются одной из основных технологией в компьютерном зрении и глубоком обучении. Появление сверточных сетей наряду с графическими процессорами (GPU), рассчитанными на массивные параллельные вычисления, позволило совершить огромный прорыв в области решения задач классификации, локализации, сегментации и т.д. 

  

\subsection{ResNet (Residual Neural Network, Остаточная нейронная сеть)} 

  

ResNet --- это популярная сверточная сеть, предложенная в работе \cite{he2016deep}. Основной особенностью являются skip свзязи --- переходы между слоями, не являющимися соседними к друг другу и помогающие в борьбе с проблемой затухающего градиента. 

  

\subsection{Кодировщик-декодировщик (Encoder-Decoder)} 

  

Encoder-Decoder является распространенной архитектурой нейронных сетей, подразумевающий наличие кодировщика и декодировщика. Кодировщик --- это последовательность слоев сети, которые преобразуют вход, существенно сокращая его размерность. На выходе кодировщика получается некоторый тензор из латентного пространства, который должен эффективно представлять вход сети, используя для этого данные меньше размерности. К выходу кодировщика применяется декодировщик, который в свою очередь преобразует латентное представление входа сети в некоторые данные большей размерности. 

  

\subsection{Region Proposal Network (RPN)} 

  

Впервые RPN была предложена в работе \cite{ren2015faster} как часть сети Faster-R-CNN и предназначается для решения задачи локализации. RPN надстраивается над выходом сверточной сети, содержащем feature map исходного изображения. Помимо наличия основного входа RPN параметризуется также и набором анкоров - участков изображения, построенных на основании размерностей исходного изображения. Для каждого из анкоров RPN необходимо подать на выход предсказание, насколько этот анкор совпадает с локализируемыми участками, и набор несколько преобразованных координат анкора. 

  

\subsection{Squeeze and Excitation (SE)} 

  

Модуль SE был предложен в 2017 году в работе \cite{hu2018squeeze} и показал SOTA результат в соревновании ImageNet. Данный модуль, который предполагается включать в сеть после каждого сверточного блока, позволяет использовать зависимости между различными каналами блока и масштабировать с помощью вектора коэффициентов, обучаемых в небольшой побочной сети. 

  

\section{Генеративные противоборствующие сети} 

  

Генеративные противоборствующие сети эффективно используются не только для генерации реалистичных фотографий, лиц и т.д., которые можно было бы использовать вместо настоящих фотографий для некоторых конечных целей, но также широко применяются для аугментации данных, которые предполагается использовать для обучения. 

  

Основная идея GAN состоит в создании двух независимых сетей --- генератора и дискриминатора, и задачей генератора является генерация реалистичных объектов по некоторому входу из латентного пространства, а задачей дискриминатора --- классификация объектов на генерированные и оригинальные. 

  

Обучение GAN представляет из себя минимаксную игру, где генератор стремится увеличить функцию потерь дискриминатора, а дискриминатор --- напротив, уменьшить ее. 

  

Также выделяют отдельный класс генеративных противоборствующих сетей --- Deep Convolutional GAN (DCGAN), которые используют глубокие сверточные сети в качестве моделей генератора и дискриминатора. 

  

\subsection{Условные генеративно-состязательные сети (Conditional GAN, CGAN)} 

  

В данной вариации GAN на вход генератору подаются не только данные из латентного пространства, но и некоторые данные условия (condition). Таким образом, генератору приходится подстраивать генерируемые данные под это условие, чтобы они выглядели реалистично. 

  

\subsection{Wasserstein GAN (WGAN) \cite{arjovsky2017wasserstein}} 

  

Данная модификация GAN заключается в том, что вместо бинарного классификатора дискриминатор выдает вещественные числа (чем больше, тем больше уверенность в оригинальности изображения). Характерной особенностью WGAN является функция потерь - Wasserstein Loss. 

  

\begin{equation} 

    L(y_{true}, y_{pred}) = mean(y_{true}y_{pred}) 

\end{equation} 

  

  

\section{Адаптивная нормализация объектов} 

  

Адаптивная нормализация объектов --- технология нормализации выходов внутренних слоев сети, которая была предложена в работе \cite{huang2017arbitrary} для совершенствования модели переноса стиля одного изображения на другое 

  

Это аффинное преобразование входа $x$ параметризованное некоторым $y$: 

  

\begin{equation} 

AdaIN(x, y) = \sigma(y)(\dfrac{x - \mu(x)}{\sigma(x)}) + \mu(y) \label{eq:adain} 

\end{equation} 

  

Авторы предполагали использовать стиль в качестве $y$, однако в последствии данную нормализацию стали применять. 

  

В области генеративных противоборствующих сетей AdaIN можно применять не только для решения задачи переноса стиля, но и для обучения эффективных параметров нормализации. В роли $y$ может выступать выход некоторой побочной сети, принимающей на вход латентный вектор. 

  

\section{Существующие решения} 

  

Рассмотрим некоторые существующие решения задач локализации и аугментации данных. 

  

\subsection{Локализация} 

  

Множество моделей локализации используют двухэтапную схему \cite{girshick2014rich} на подобии следующей:  

  

\begin{enumerate} 

    \item Получения region proposals из входного изображения и вычисление признаков для каждого из proposal-ов с помощью 3D-CNN 

    \item Классификация участков (в нашем случае на содержащие и не содержащие опухоль) 

\end{enumerate} 

  

Основной причиной выбора такой двухэтапной схемы является большое количество ложно положительных предсказаний, получающихся в end-to-end методах, которые возможно устранить на втором этапе подобной схемы (FP reduction). 

  

Другой класс моделей использует end-to-end 3D-CNN для решения задачи локализации. В частности популярна сеть 3D-Faster-RCNN, включающая с себя RPN. 

  

\subsection{Аугментация} 

  

В ряде работ предложена аугментация датасетов с помощью. В частности в работе \cite{wgan-augmentation} использовался WGAN, а в работе \cite{han2019synthesizing} использовался Multi-Conditional GAN. В качестве модели локализации (для обучения которой и необходима аугментация) использовалась Faster-RCNN, а в качестве GAN --- MCGAN c 4-мя сверточными слоями в кодировщике и 4-мя в декодировщике, а также skip-связями. Дискриминатор по своей структуре похож на Pix2Pix GAN, а в качестве функции потерь использовался Wasserstein Loss (с Gradient Penalty) наряду с Least Squares Loss. 

  

\section{Постановка цели и задач ВКР} 

  

Целью работы является разработка методов и программных средств автоматического распознавания новообразований в легких на изображениях компьютерной томографии.  

  

Среди задач можно выделить два подхода, альтернативных друг другу 

  

\begin{enumerate} 

    \item Сегментация двухмерных изображений для определения маски, соответствующей опухоли 

    \item Локализация трехмерного участка, содержащего новообразование, на трехмерном изображении. 

\end{enumerate} 

  

Отметим, что вторая задача является существенно более распространенной несмотря на то, что 

привычным методом поиска новообразований является сегментация изображений. В данном же случае локализация подходит лучше, поскольку опухоли достаточно компактны и малы, и естественно предсказывать их наличие и местоположение с помощью bounding box. 

  

При рассмотрении второй задачи была сформулирована еще одна подзадача, которая может существенно помочь в решении задачи локализации, --- аугментация данных. Размеры используемых датасетов невелики, несмотря на их громоздкость, и генерация искусственных данных может существенно улучшить качество работы моделей локализации. 

  

Эффективность использования аугментации данных с помощью GAN подтверждена несколькими работами в том числе \cite{han2019synthesizing} \cite{wgan-augmentation-2} \cite{wgan-augmentation} 


\section{Выводы по главе 1}

В данной главе приведены и кратко описаны основные базовые методы и технологии, заимствованные в работе или имеющие к ней отношение. Наряду с обзором предметной области в данной главе описаны цель и задачи, поставленные в данной работе, а так же их обоснование.

\finishrelatedwork

\chapter{Решение задачи 2D сегментации} 

  

\section{Распространенный подход} 
  

Было проведено множество исследований в области сегментации КТ изображений легких. Одним из основных подходов является двух-шаговая модель, где на первом этапе локализуется предполагаемый 3D участок, содержащий опухоль (Volume of Interest, VOI), а на втором этапе производится непосредственная сегментация внутри выбранного участка. Интуитивное объяснение данной модели состоит в том, что пространство, занимаемое опухолью, достаточно мало по сравнению со всем КТ изображением. 

  

\section{Предлагаемый подход} 


Однако было предложено использовать другую модель, производящую сегментацию изображения end-to-end. Данная модель показала отличный результат (первое место) в соревновании Data Science Bowl 2018. В последствии модель была применена к другой задаче - сегментации глиальных опухолей головного мозга по данным МРТ, где показала хороший результат. 


Архитектура модели включает глубокую encoder-decoder сверточную сеть типа Unet. В архитектуре заимствованы несколько слоев сети Xception, представленной в работе \cite{chollet2017xception}, каждый из которых используется для создания своего feature map, тем самым реализуя метод feature pyramid network (FPN) из работы \cite{lin2017feature}.

Также модель использует комбинированную функцию потерь, состоящую из кросс-энтропии наряду с мягкой функцией потерь Дайса. Ожидалось, что модель способна показать хороший результат и в задаче сегментации КТ легких, в том числе потому что FPN свойственно неплохо решать задачу сегментации небольших объектов.


На рисунке \ref{mri-sample} визуализирована работа сети на изображении МРТ \cite{unet-mri}. 

  

\begin{figure}[!h] 

\includegraphics[width=\linewidth]{images/mri-sample.png} 

\caption{Пример работы сети на изображении МРТ \cite{unet-mri}}\label{mri-sample} 

\centering 

\end{figure} 



\section{Данные}

В качестве датасета использовался LIDC-IDRI \cite{lidc} - открытый набор данных в формате DICOM, размеченный 4-мя радиологами. Всего в датасете содержится 1018 сканов и более 7000 новообразований. Новообразование включалось в размеченное множество, если оно было выявлено хотя бы одним из специалистов.


\section{Детали реализации}

\subsection{Обучение}

Изначально планировалось подавать на вход сети полные 2D изображения для того, чтобы сеть смогла получить эффективное представление признаков изображений и выявить характерные свойства пространственного расположения опухолей. Однако результат данного подхода был отрицательный, поскольку сеть была не в состоянии обучиться и функция потерь не уменьшалась. Попытки использовать другую архитектуру Unet сети и изменить функцию потерь на функцию Focal Loss, отдающую предпочтение ложно положительным пикселям нежели ложно-отрицательным, не привели к успеху. Неудачу также потерпела попытка уменьшить количество данных в обучающем множестве убрав двухмерные изображения, не содержащие опухоли вообще, которых было более 80\%.

Подобное поведение сети скорее всего объясняется тем, что градиентный спуск застрявает в локальном минимуме, который соответствует тому, что сеть выдает пустые изображения, поскольку площадь маски очень мала по сравнению с изображеним.

Чтобы решить данную проблемы было предложено подавать сети не полные изображения, а их обрезанные куски. Изображение делилось сеткой на $n^2$ равных частей, из которых выбирались две части - содержащая опухоль и не содержащая. Далее эти части подавались на вход сети. При $n = 8$ сеть все еще не могла обучиться, но при $n = 16$ удалось получить результат.

\subsection{Тестирование}

Поскольку адекватный результат сеть выдавала только на маленьких обрезанных кусках изображения, для тестирования была использована следующая стратегия:

\begin{enumerate}
    \item Изображение делилось сеткой на $16^2$ частей
    \item Каждая часть независимо подавалась на вход сети
    \item Выходы сети склеивались для получения итоговой маски
    \item Так как выход сети представляет из себя двумерный массив пикселей, где значение каждого пикселя является действительным числом, вручную выбирается некоторый порог $0 < t < 1$
    \item Все пиксели имеющие значение, большее $t$, считаются пикселями маски, остальные пиксели зануляются
\end{enumerate}



\section{Подсчет метрик качества}

Основным показателем соответствия между истинной и предсказанной маской был выбран коэффициент Дайса

$$ dice(y_{true}, y_{pred}) = \dfrac{2|y_{true} \cap y_{pred}|}{ |y_{true}| + |y_{pred}|} $$

Если модель выдавала пустую маску, и при этом в реальности на изображении не было патологий, коэффициент Дайса полагался равным $1$. Ниже приведены коэффициенты Дайса, усредненные по тестовой выборке для различных значений порога $t$ 

В качестве примера таблицы приведена таблица~\ref{tab1}.

\begin{table}[!h]
\caption{Таблица умножения (фрагмент)}\label{tab1}
\centering
\begin{tabular}{|*{18}{c|}}\hline
\textbf{Порог $t$} & \textbf{Коэффициент Дайса} \\\hline
0.4 & 0.15 \\\hline
0.45 & 0.08 \\\hline
0.5 & 0.1 \\\hline
\end{tabular}
\end{table}

\section{Визуальный анализ}

\begin{figure}[!h]
\includegraphics[width=\linewidth]{images/2d-seg-results.png}
\caption{Результаты работы сети}\label{mirskiy-cgan-architecture}
\centering
\end{figure}

Кругом обозначены совпадения на выходе сети и на истинной маске. На нижнем примере видно, что сеть ошибочно распознала участок ткани визуально похожий на опухоль, но в реальности ею не являющийся.

\section{Выводы по главе 2}

Полученные результаты явно свидетельствуют о непригодности использования подобных $end-to-end$ технологий для сегментации $2d$ изображений. Предполагаемое объяснение состоит в том, что двухмерные сканы не в состоянии передать достаточный набор признаков, позволяющий отличить опухоль от другого, похожего на нее участка легкого. Дело в том, что каждый двухмерный слайс содержит только лишь срез трехмерной опухоли, и значительная часть таких срезов может иметь вполне неопределенную форму, ничем не отличимую от сосудов в легких. Отсюда следует большая необходимость в использовании трехмерных зависимостей между различными слайсами для решения задачи распознавания.



\chapter{Описание предлагаемого подхода для 3D локализации}

В этой главе будут описан предлагаемое решение задачи 3D локализации новообразований по снимкам КТ легких.

\section{Модель локализации опухолей}

Для детекции опухолей была заимстована сверточная сеть с архитектурой Squeeze and Excitation Encoder Decoder \cite{li2019deepseed}. Сеть использует архитектуру ResNet18 в качестве кодировщика со встроенными SE модулями после каждого сверточного блока ResNet (остаточного блока). 

Структуру кодировщика отражает декодирощик, слои которого связаны skip-соединениями с кодировщиком. Выход декодирвщика подается на вход Region Proposal Network (RPN), задачей которой является предсказание координат различных bounding box-ов и вероятностей того, что в том или ином bounding box-е заключено новообразование.

\begin{figure}[!h]
\includegraphics[width=\linewidth]{images/deep-seed-architecture.png}
\caption{Архитектура используемой сети DeepSEED}\label{deep-seed-architecture}
\centering
\end{figure}


\section{Аугментация данных с помощью генеративных противоборствующих сетей}

Широко известно применение генератиных противоборствующих сетей для генерации данных, которые впоследствии могут быть использованы в качестве расширения существующего датасета, на котором предполагается обучать детектор. В ряде работ были проведены визуальные тесты Тьюринга, где радиологам было предложено попробовать отличить генерированные опухоли от оригинальных. Специалисты не очень хорошо справились с задачей, что может говорить о том, что сети способны генерировать достаточно хорошие изображения, чтобы их можно было использовать в обучающем множестве.

\subsection{Модель генеративной противоборствующей сети}

Для решения данной задачи было принято решение заимствовать сеть, предложенную в работе \cite{mirsky}. У данной статьи есть документированная и удобная реализация в открытом доступе, в которой реализованы два фреймворка - один предназначен для добавления генерированных опухолей на изображение, второй предназначем для удаления опухолей с изображения. Для задачи аугментации данных естественно позаимствовать первый фреймворк который помимо реализации непосредственно архитектуры сети, предоставляет еще и возможность предобработки данных, которая состоит из нормализации и гистограммного выравнивания изображений.

Модель построена на условных GAN (Conditional GAN, CGAN), и основной задачей генератора является создание правдоподобной опухоли из некоторого шума и контекста. Сеть получает на вход $x_r^*$ - участок КТ изображения размера $32^3$, содержащий опухоль, из которого вырезан центральный куб размера $16^3$, и на его место вставлена маска из нулей. Оставшаяся часть участка $x_r^*$ выступает в роли контекстного окружения. Сеть генератора состоит из нескольких сверточных слоев (кодировщика) и нескольких декодирующих слоев, которые соединены с кодирующими слоями попарно (skip connection). После каждого сверточного блока применяется батчевая нормализация. Выход генератора - $G(x_r^*, \theta_g)$ наряду с оригинальным участком изображения $x_r$ далее подаются на вход дискриминатору, задачей которого является классифицировать объекты как оригинальные или генерированные.

\begin{figure}[!h]
\includegraphics[width=\linewidth]{images/mirskiy-cgan-architecture.png}
\caption{Архитектура используемой сети CGAN}\label{mirskiy-cgan-architecture}
\centering
\end{figure}

\subsection{Адаптивная нормализация объектов (AdaIN)}

Адаптивная нормализация является популярной технологией и служит альтернативой батчевой нормализации и объектной нормализации, применяемой после каждого сверточного блока. Поэтому для совершенствования генеративной сети было предложено использовать ее вместо батчевой нормализации для эффективного и обучаемого преобразования промежуточных данных.

Адаптивная нормализация применяется следующим образом: Тензором $x$ принимается на вход. Размерности тензора: 

$$(b, xdim, ydim, zdim, c)$$, где $b$ - количество объектов в батче, $c$ - количество каналов, $xdim, ydim, zdim$ - пространственные размерности сооответственно.

нормализуется по все пространственным измерениям, то есть это происходит независимо для каждого канала и объекта в отличие от батчевой нормализации и аналогично объектной нормализации (Instance Normalization). Параметры афинного преобразования имеют размерность 

$$(b, 1, 1, 1, c)$$

То есть данные параметры аналогично варьируются по каналам и объектам.

\section{Итоговая процедура обучения}

В результате предложенная процедура обучения имеет следующий вид:

\begin{enumerate}
    \item Из датасета КТ изображений легких выбираются те, которые удовлетворяют некоторым ограничениям и соответственно являются пригодными для обучения CGAN
    \item Из выбранных данных вырезанются участки, содержащие опухоль, для создания объектов, которые можно непосредственно подать на вход CGAN.
    \item Данные аугментируются стандартными способами (повороты, отражения)
    \item Данные проходят предобработку: гистограммное выравнивание и нормализацию
    \item Данные, снабженные шумом на месте опухоли и окружающим контекстом подаются на вход CGAN для обучения
    \item Сеть CGAN, натренированная генерировать опухоли получает на вход предобработанные контексты для непосредственной генерации
    \item Данные, полученные на выходе CGAN, дополняют оригинальный датасет
    \item Сеть DeepSEED обучается на дополненном датасете.
\end{enumerate}



\chapter{Результаты решения задачи 3D локализации}

\section{Используемые данные}

В качестве датасета использовался не LIDC-IDRI непосредственно, как это было сделано в решении задачи 2d-сегментации, а его подможество LUNA \cite{luna}, содержащее 888 сканов в формате MetalImage и 1187 опухолей, каждая из которых представленна в виде bounding box. В отличие от LIDC-IDRI из LUNA исключены сканы, по толщине превосходящие 2.5 мм, а также включены только те опухоли, которые были размечены хотя бы тремя и четырех специалистов-радиологов. Еще одна особенность LUNA по сравнению с LIDC состоит в том, что средний размер новообразований в LUNA составляет 8.3 мм при стандартном отклонении в 4.8 мм, когда для LIDC-IDRI эти показатели составляют 12.8мм и 10.6 мм соответствено.

\section{Детали реализации}

\subsection{Реализация DeepSEED}

В оригинальной работе было предложено подавать на вход сети обрезанные участки КТ изображений размера $128^3$, однако такая размерность входных данный требует вычислительных ресурсов, которые были недоступны во время реализации, что привело к уменьшению размерности входа до $64^3$.


\subsection{Реализация Адаптивной нормализации}

Адаптивная нормализация была добавлена в сеть вместо батчевой нормализации. Она применяется после всех блоков (свертка, активация) и в кодировщике, и в декодировщике кроме первого слоя. В качестве входа $x$ формулы \eqref{eq:adain} используется тензор, полученный на выходе активации, а в качестве параметра $y$ используется выход побочной сети, имеющей простую архитектуру - 3 полносвязных слоя, причем первые два разделены между всеми побочными сетями. Данная архитектура позволяет выявить эффективные параметры афинного преобразования посредством обучения.

\subsection{Реализация CGAN}

Изначально архитектура CGAN, формат и размерность входных и выходных данных были полностью заимствованы у авторов \cite{mirsky} в качестве бейслайн решения. Отметим, что при обучении генератора авторы использовали не стандартную функцию потерь, а комбинированную:

\begin{equation}
L(o, g, D_{output}, labels) = MSE(o, g) + MAE(o, g) + L_{G}(D_{output}, labels)
\end{equation}

$o, g$ - оригинальное и генерированное изображения, соответствующие одному контексту.

$L_{GAN}$ - это стандартная функция потерь генератора, рассчитываемая на выходах дискриминатора и реальных метках объектов. Посредством уменьшения данной функции потерь генератор непосредственно стремится обмануть диксриминатор. Однако наряду с этим авторы предлагают использовать функцию потерь, минимизирующую различия между оригинальными изображениями и генерированными.

Я обучил несколько моделей с полностью заимствованными параметрами, но во время обучения и оценки результатов было обнаружено несоклько проблем:

\begin{enumerate}
    \item Функция потерь дискриминатора оптимизировалась гораздо лучше, чем функция потерь генератора, и начиная с некоторой эпохи, дискриминатор почти всегда отличал генерированные изображения от оригинальных.
    
    \item Выходные изображения были достаточно четкими в области контекста, однако в области латентного входа, где предполагалось генерировать непосредственно опухоль, изображения были мутные.
    
    \item Некоторые модели генерировали очень похожие результаты на различных входах. Данная проблема известна как mode collapse.
\end{enumerate}

Для борьбы с данными проблемами было предпринято несколько различных модификаций сети

\begin{enumerate}
    \item Вариация соотношения числа обновлений генератора и дискриминатора
    \item Увеличение размерности входа 
    \item Вариация функции потерь
\end{enumerate}

\subsubsection{Вариация соотношения числа обновлений генератора и дискриминатора}

Данный подход является стандартным для решения ситуации, когда либо генератор, либо дискриминатор отстает от другого и обучается существенно медленнее, либо совсем не обучается, и функция потерь даже возрастает. В моделях бейслайн архитектуры функции потерь и генератора и дискриминатора уменьшались, однако точность дискриминатора достигала 100\% и далее не уменьшалась. Несмотря на то, что генератору не удавалось обмануть дискриминатор, обучениие продолжало происходить в связи с присутствием $MSE$ и $MAE$ компонент. Данный подход кажется неудовлетворительным, поскольку в таком случае роль дискриминатора фактически отпадает и генератор просто стремится произвести похожее на оригинал изображение.

При увеличении соотношения числа обновлений с $1:1$ до $10:1$ в пользу генератора удалось получить визуально лучшие результаты представленные на рисунке \ref{cgan-10wu-no-adain-baseline-loss}

\begin{figure}[!h]
\includegraphics[width=\linewidth]{images/gan-results/no-adain.png}
\caption{Пример генерированных изображений (Без AdaIN)}\label{cgan-10wu-no-adain-baseline-loss}
\centering
\end{figure}

\subsubsection{Увеличение размерности входа}

Для избавления от проблемы mode collapse зачастую применяется увеличение латентного пространства. Если латентное пространство недостаточно велико, то модель физически не в состоянии генерировать хорошие и разнообразные изображения. Mode collapse - это состояние, когда модель отдает предпочтение одному или нескольким шаблонам и периодически меняет их, если дискриминатор подстраивается под эти шаблоны. В данном случае, так как мы используем СGAN, необходимо увеличить размерность контекста, оставляя нулевую маску в центре куба такой же.

Были предприняты попытки увеличить размерность кропов до $48^3$, то есть размерность латентного пространства увеличилась с $32^3 - 20^3$ до $48^3 - 20^3$, что кажется достаточно значительным изменением.

Однако результаты модели получились отрицательные и визуально были существенно хуже, чем для размера кропов $32^3$

\subsubsection{Wasserstein Loss}

Данный подход призван решить проблему максимальной точности, достигаемой дискриминатором. Предлагается добавить в функцию потерь Wasserstein Loss. Архитектура была настроена соответствующим образом: дискриминатор был преобразован в критика изменением выходного значения с бинарной метки на вещественное число. Тем не менее компоненту функции потерь - $MSE(original, generated)$ было решено оставить, поскольку экспериментально полученные результаты были лучше при ее наличии.

На рисунке \ref{loss} представлено изменение функции потерь в зависимости от количества обновлений генератора или критика. Заметим, что при обучении генератор обновлялся в 5 раз больше, чем критик. На рисунке видно, что генератору в целом удается обмануть критика, (в отличие от бейслайн модели), и при этом обучение сходится.

\begin{figure}[!h]
\includegraphics[width=\linewidth]{images/gan-injections/1.png}
\caption{Loss changes over time for WGAN}\label{loss}
\centering
\end{figure}

\subsubsection{Итоговая модель}
\label{section:final-gan}
В итоговую процедуру обучения вошли

\begin{enumerate}
    \item Эвристика, поддерживающая соотношение между обновлением генератора и дискриминатора, равное $5:1$ соответственно.
    \item Wasserstein Loss, комбинированный с MSE(generated, original)
    \item AdaIN (включен в модель)
\end{enumerate}

Пример генерированных итоговой моделью изображений показан на рисунке \ref{cgan-final}. На рисунке \ref{mirsky-results} представлены результаты из оригинальной работы \cite{mirsky}

\begin{figure}[!h]
\includegraphics[width=\linewidth]{images/gan-results/final-gan.png}
\caption{Пример генерированных итоговой моделью изображений}\label{cgan-final}
\centering
\end{figure}

\begin{figure}[!h]
\includegraphics[width=\linewidth]{images/mirskiy-results.jpg}
\caption{Пример генерированных изображений из работы \cite{mirsky}}\label{mirsky-results}
\centering
\end{figure}

\subsection{Реализация аугментации LUNA}

Для аугментации непосредственного датасета LUNA использовалось 366 сканов из тренировочного множества, в котором всего 800 сканов, то есть процент аугментированных данных составил $31$ в итоговом экмпериментальном запуске обучения локализатора. Для аугментации использовалась модель, представленная в \ref{section:final-gan}. Участки тканей, содержащие опухоль вырезались, проходили предобработку, подавались на вход генератору, выход генератора проходил постобработку и далее вырезанные участки вставлялись обратно в скан. Процесс инъекии генерированных участков был полностью позаимствован из \cite{mirsky}.

Обратим внимание на то, каким образом выбирались участки для инъекции: было принято решение выбирать участки, содержащие настояющую опухоль. Это было сделано по нескольким причинам:

\begin{enumerate}
    \item Таким образом не нарушаются пространственные признаки участков опухоли в используемых данных (то есть мы не будем вставлять опухоль туда, где она биологически маловероятно может появиться)
    \item Поскольку мы использовали WGAN, а не оригинальную модель, генерируемые опухоли в гораздо меньшей степени напоминают оригинальные при таком же контексте, то есть разнообразие и случайность опухолей достаточно велики
    \item Обнаружение мест, пригодных для инъекции, не является простой задачей и требует либо решения дополнительной задачи сегментации и валидации ее результатов, либо ручной разметки.

\end{enumerate}


На рисунке \ref{injection} показан пример генерированной опухоли, вставленной в скан, по сравнению с оригинальной. Стоит упомянуть, что так как обучение GAN происходило на отобранных опухолях диаметра не меньше 10мм и не больше 16мм, размер генерированной опухоли несколько больше, чем оригинальной.

\begin{figure}[!h]
\includegraphics[width=\linewidth]{images/gan-injections/1.png}
\caption{Пример опухоли, вставленной в скан}\label{injection}
\centering
\end{figure}

\section{Результаты}

\subsection{Результаты локализации}

Оценка качества моделей производится с помощью несколько измененного FROC анализа, представленного в официальном скрипте LUNA Challenge \cite{luna} и широко используемого в работах по локализации новообразований в легких (в частности в работе \cite{li2019deepseed}). Оценка модели представлена набором точек, формирующих FROC кривую, где каждая точка представляет среднее количество ложно положительных предсказаний на один скан (average fp / scan) - по оси x - и соответствующую ему чувствительность - по оси y. Отметим, что классический FROC-анализ не был произведен, поскольку в методе оценки (average fp  / scan) \cite{luna} модели предполагается подавать на вход кропы большого  размера, а именно $208^3$, что невозможно в рамках текущей работы по причине недостатка вычислительных мощностей.

В связи с этой проблемой было решено произвести аналогичный анализ, но по оси x будут отмечены средние значения количества ложно положительных предсказаний на один кроп размера $64^3$. Уместность данного анализа обоснована тем, что каждая точка на оси x соответствует некоторому значению порога модели, контролирующего среднее число ложно-положительных предсказаний, которые склонна находить модель. Так как основной задачей работы является аугментация данных с помощью GAN, интересно в первую очередь оценить относительный выигрыш в локализации при использовании аугментированных данных в обучении.

Приведем метрики для полученных моделей в виде таблицы \ref{tab:result-metrics} и графика \ref{image:final-results}


\begin{table}[!h]
\caption{Численные результаты}\label{tab:result-metrics}
\centering
\begin{tabular}{|*{18}{c|}}\hline
\textbf{Модель} & \textbf{0.25} & \textbf{0.5} & \textbf{1} & \textbf{2} & \textbf{4} & \textbf{8} \\\hline
Без аугментации, 100 эпох & 0.44 & 0.56 & 0.65 & 0.78 & 0.85 & 0.87 \\\hline
С аугментацией, 100 эпох & 0.48 & 0.58 & 0.71 & 0.81 & 0.88 & 0.9 \\\hline
\end{tabular}
\end{table}

\begin{figure}[!h]
\includegraphics[width=\linewidth]{images/result_plot.png}
\caption{Итоговые FROC (over crop) кривые}\label{image:final-results}
\centering
\end{figure}


\begin{figure}[!h]
\includegraphics[width=\linewidth]{images/gan-results/adain.png}
\caption{Пример генерированных изображений (С использованием AdaIN)}\label{cgan-adain-results}
\centering
\end{figure}

\subsection{Сравнение результатов с другими работами}

Проведем сравнение с работой \cite{han2019synthesizing}, в которой были произведена аналогичная аугментация, однако использовался LIDC-IDRI вместо LUNA, в качестве модели локализации использовалась Faster-RCNN, а в качестве GAN - MCGAN c 4-мя сверточными слоями в кодировщике и 4-мя в декодировщике, а также skip-связями. Дискриминатор по своей структуре похож на Pix2Pix GAN, а в качестве функции потерь использовался Wasserstein Loss (с Gradient Penalty) наряду с Least Squares Loss. На вход подавались данные размера $64^3$, где середина размера $32^3$ заменялась случайным шумом, что существенно увеличивает размерность латентного пространства по сравнению с нашей моделью.

На рисунке \ref{han-froc-plot} приведены графики, а на рисунке \ref{han-cpm} численные результаты. CPM (Competition Performance Metric) представляет из себя усредненную чувствительность по (0.125, 0.25, 0.5, 1, 2, 4, 8) значениям average fp / scan.

\begin{figure}[!h]
\includegraphics[width=\linewidth]{images/mcgan-results.png}
\caption{График FROC-кривой из работы \cite{han2019synthesizing}}\label{han-froc-plot}
\centering
\end{figure}

\begin{figure}[!h]
\includegraphics[width=\linewidth]{images/han-cpm.png}
\caption{Численные результаты работы \cite{han2019synthesizing}}\label{han-cpm}
\centering
\end{figure}

%% Макрос для заключения. Совместим со старым стилевиком.
\startconclusionpage

В данном разделе размещается заключение.

\section{Выводы}

В рамках работы не удалось решить задачу 2-d сегментации по причине неприменимости подобных 2d end-to-end методов к датасету LIDC-IDRI. Это еще более подтвердило уместность задачи локализации. Заимствованную модель для решения задачи локализации удалось обучить на кропах меньшего размера и получить относительно неплохие показатели чувствительности.

Оптимизированная архитектура GAN, предложенная в работе \cite{mirsky}, показала свою способность улучшить качество локализатора и порождать правдоподобные и разнообразные опухоли посредством добавления Wasserstein Loss и модуля Adain.

\section{Применимость полученных методов на современных данных}

Современные томографы обладабт более высоким разрешением, нежели сканы, представленые в использованных датасетах. Пространственное разрешение может быть более, чем в 4 раза выше по каждой оси, что ведет к огромному весу полных 3d сканов. Однако предложенные методы способны обрабатывать входные данные неограниченных размеров, и модель локализации при необходимости можно обучить на входных данных большего размера

Существует несколько способов решения этой проблемы:

\begin{enumerate}
    \item Уменьшение размерности КТ изображений
    \item Обучение модели локализации и аугментации на данных большего разрешения. Возомжно при достаточных вычислительных мощностях
\end{enumerate}

\printmainbibliography

\appendix

\chapter{Графики FROC метрик}\label{sec:app:1}

В данном приложении приведены пять графиков FROC-метрик для сравнительного ананлиза бейслайн решения и предложенной модификации с аугментацией. Каждый график соответствует одному эксперименту из таблицы \ref{tab:result-metrics}

\begin{figure}[!h]
\includegraphics[width=\linewidth]{images/froc-results/cv_plot_0.png}
\caption{Итоговые FROC (over crop) кривые}\label{image:final-results-0}
\centering
\end{figure}

\begin{figure}[!h]
\includegraphics[width=\linewidth]{images/froc-results/cv_plot_1.png}
\caption{Итоговые FROC (over crop) кривые}\label{image:final-results-1}
\centering
\end{figure}

\begin{figure}[!h]
\includegraphics[width=\linewidth]{images/froc-results/cv_plot_2.png}
\caption{Итоговые FROC (over crop) кривые}\label{image:final-results-2}
\centering
\end{figure}

\begin{figure}[!h]
\includegraphics[width=\linewidth]{images/froc-results/cv_plot_3.png}
\caption{Итоговые FROC (over crop) кривые}\label{image:final-results-3}
\centering
\end{figure}

\begin{figure}[!h]
\includegraphics[width=\linewidth]{images/froc-results/cv_plot_4.png}
\caption{Итоговые FROC (over crop) кривые}\label{image:final-results-4}
\centering
\end{figure}

\end{document}