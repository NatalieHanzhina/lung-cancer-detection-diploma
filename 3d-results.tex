\chapter{Результаты решения задачи 3D локализации}

\section{Используемые данные}



\section{Детали реализации}

\subsection{Реализация DeepSEED}

В оригинальной работе было предложено подавать на вход сети обрезанные участки КТ изображений размера $128^3$, однако такая размерность входных данный требует вычислительных ресурсов, которые были недоступны во время реализации, что привело к уменьшению размерности входа до $64^3$.

\subsection{Реализация CGAN}

Основная архитектура CGAN, формат и размерность входных и выходных данных были практически полностью заимствованы у авторов статьи.

\subsection{Реализация Адаптивной нормализации}

Адаптивная нормализация была добавлена в сеть вместо батчевой нормализации. Она применяется после всех блоков (свертка, активация) и в кодировщике, и в декодировщике кроме первого слоя. В качестве входа $x$ формулы $kek$ используется тензор, полученный на выходе активации, а в качестве параметра $y$ используется выход побочной сети, имеющей простую архитектуру - 3 полносвязных слоя, причем первые два разделены между всеми побочными сетями. Данная архитектура позволяет выявить эффективные параметры афинного преобразования посредством обучения.

\section{Результаты}

\subsection{Результаты локализации с помощью DeepSEED}

