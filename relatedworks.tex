\section{Сверточные нейронные сети}

\section{ResNet}

\section{Squeeze and Excitation}

Squeeze and Excitation (SE)

Модуль SE был предложен в 2017 году и показала SOTA результат в соревновании ImageNet. Данный модуль, который предполагается включать в сеть после каждого сверточного блока, позволяет использовать зависимости между различными каналами блока и масштабировать с помощью вектора коэффициентов, обучаемых в отдельной маленькой сети (какой??) 

[картинка]


\section{Генеративные противоборствующие сети}


\section{Адаптивная нормализация объектов}

Адаптивная нормализация объектов - технология нормализации выходов внутренних слоев сети, которая была предложена в работе[] для совершенствования модели переноса стиля одного изображения на другое

Это афинное преобразование входа $x$ параметризованное некоторым $y$:

$$AdaIN(x, y) = \sigma(y)(\dfrac{x - \mu(x)}{\sigma(x)}) + \mu(y)$$

Авторы предполагали использовать стиль в качестве $y$, однако в последствии данную нормализацию стали применять.

В области генеративных противоборствующих сетей AdaIN можно применять не только для решения задачи переноса стиля, но и для обучения эффективных параметров нормализации. В роли $y$ может выступать выход некоторой побочной сети, принимающей на вход латентный вектор.